\subsection{Prinzipien}
\label{subsec:Prinzipien}
\subsubsection{Annotationen}
\label{subsubsec:Annotationen}
Zur Kompilierung von \texttt{Spark} wird ein \texttt{Ada}-Kompiler verwendet, das heißt jedes \textit{gültige} Programm in \texttt{Spark} ist auch in \texttt{Ada} gültig. Es gibt jedoch viel Dinge , die in \texttt{Spark} nicht gültig sind, jedoch in \texttt{Ada}. \texttt{Spark} ist somit eine Teilmenge von \texttt{Ada}, ergänzt um zusätzliche Annotationen mit zusätzlichen Angaben für die Prüfungstools von \texttt{Spark}. Da diese Annotationen als Kommentare ausgeführt sind, bleibt durch sie ein Programm immer noch in \texttt{Ada} gültig, werden vom Kompiler aber nicht verarbeitet, sondern sind nur für den \textit{Examiner} von \texttt{Spark} wichtig.\\

\subsubsection{Unterprogramme}
\label{subsubsec:Unterprogramme}

Listing~\ref{spark:derive} zeigt ein Beispiel der Annotation \textit{derives}, welche dem Examiner zeigt, dass der am Ende der Variablen X zugewiesenen Wert vom Initialwert der Variablen Y abhängt und umgekehrt.
\begin{lstlisting}[caption={derive Beispiel}, label=spark:derive]
procedure Exchange(X,Y:in out Float)
--# derives X from Y &
--# 				Y from X;
is
	T:Float;
begin
	T := X; X := Y; Y := T;
end Exchange;
\end{lstlisting}

Listing~\ref{spark:GlobaleVariablen} zeigt die Nutzung der globalen Variablen Seed. Die Annotation \textit{own} macht die Variable für andere Annotationen sichtbar, \textit{initializes} zeigt an, dass sie Initialisisert wird.


\begin{lstlisting}[caption={Globale Variablen}, label=spark:GlobaleVariablen]
package RandomNumbers
--# own Seed;
--# initializes Seed;
is
	procedure Random(X: out Float);
	--#global in out Seed;
	--# derives X,Seed from Seed;
end RandomNumbers;

package body RandomNumbers
	Seed:Integer;
	SeedMax: constant Integer:= ...;
	
	procedure Random(X: out Float) is
	begin
		Seed:= ...;
		X := Float(Seed) / Float(SeedMax);
	end Random;
	
begin		--Initialization part
	Seed:= 12345
end RandomNumbers;
\end{lstlisting}