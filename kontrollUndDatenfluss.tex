\subsection{Kontroll- und Datenfluss}
\label{subsec:KontrollUndDatenfluss}
\textit{if}- und \textit{case}-Strukturen sind in \texttt{Spark} genauso wie in \texttt{Ada}. \textit{for}-Schleifen sind leicht anzupassen, s. Listing~\ref{spark:for-Schleifen}.
\begin{lstlisting}[caption={for-Schleifen}, label=spark:for-Schleifen]
for I in 1 .. 10 loop  --falsch

for I in Integer range 1 .. 10 loop
\end{lstlisting}

\textit{while}- und \textit{for}- Schleifen ebenso wie einfache \textit{loop}-Anweisungen können auch \textit{exit}-Anweisungen enthalten. Auch mehrere exit-Anweisungen in einem \textit{loop} werden immer alle ausgewertet.
\begin{lstlisting}[caption={exit in loop}, label=spark:exit in loop]
loop
	Get(CurrentCharacter);
	exit Search when CurrentCharacter = '*';
end loop Search;
\end{lstlisting}

\textit{return}-Anweisungen können in \texttt{Spark} nur in Funktionen verwendet werden und das auch nur exakt einmal am ende der Funktion. Es ist nicht möglich sie wie in \texttt{Ada} zum zurückkehren zum Prozeduraufruf zu verwenden.\\\


Es gibt in \texttt{Spark} keine \textit{Exception}, \textit{Labels}, \textit{Goto}-Anweisungen, \textit{Blöcke}. Unterprogramme können nicht rekursiv aufgerufen werden.